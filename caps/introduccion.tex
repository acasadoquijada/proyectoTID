En este primer apartado vamos a hablar un poco sobre el dataset que hemos elegido para realizar nuestro trabajo.

Este dataset contiene datos recogidos de la aplicación PokemonGo, esta aplicación es un juego de realidad aumentada que emplea el GPS del móvil para principalmente localizar y capturar pokemon en el mundo real. El dataset contiene 296021 muestras cada una de las cuales dispone de los siguientes campos:

\begin{itemize}

\item \textbf{pokemonId}: el identificador del pokemon, denota su clase.

\item \textbf{latitude}: latitud de la posición donde se ha localizado el pokemon.

\item \textbf{longitude}: longitud de la posición donde se ha localizado el pokemon.

\item \textbf{appearedLocalTime}: momento exacto en el que se encontró el pokemon, con el formato yyyy-mm-ddThh-mm-ss.ms.

\item \textbf{X\_id}: la ficha del dataset no proporciona información sobre qué representa este dato, de hecho hemos \href{https://www.kaggle.com/semioniy/predictemall/discussion/24061#144095}{preguntado en la propia web}) pero no hemos obtenido respuesta. No obstante viendo el dataset no parece ser más que una código identificador de la muestra. 

\item \textbf{cellId\_90-5850m}: la localización goegráfica del pokemon proyectada en una celda S2 de distinto tamaño.

\item \textbf{appearedTimeOfDay}: momento del día en el que apareció el pokemon (night, evening, afternoon, morning).

\item \textbf{appearedHour}: hora local de una observación del pokemon.

\item \textbf{appearedMinute}: minuto local de una observación del pokemon.

\item \textbf{appearedDayOfWeek}: día de la semana en la que se produjo el avistamiento (Monday, Tuesday, Wednesday, Thursday, Friday, Saturdy, Sunday).
 
\item \textbf{appearedDay}: día del avistamiento.

\item \textbf{appearedMonth}: mes del avistamiento.

\item \textbf{appearedYear}: año del avistamiento.

\item \textbf{terrainType}: tipo del terreno donde se avistó el pokemon. Este dato viene dado por un valor número según una \href{http://glcf.umd.edu/data/lc}{tabla de tipos de terreno}

\item \textbf{closeToWater}: si el avistamiento se produjo a 100m o menos del agua o no.

\item \textbf{city}: pese a que en la página del dataset nos dice que se trata de la ciudad donde se produjo el avistamiento más adelante veremos que esto no así y daremos nuestra interpretación a este atributo así como al de continent.
 
\item \textbf{weather}: un string indicando el tiempo que hacía en el momento del avistamiento.
 
\item \textbf{temperatureç: temperatura en grados Celsius en el momento del avistamiento.

\item \textbf{windSpeed}: velocidad del viento en el momento del avistamiento km/h.

\item \textbf{windBearing}: dirección del viento entre 0 y 360 grados.

\item \textbf{pressure}: presión en el momento del avistamiento en bares.

\item \textbf{weatherIcon}: el tiempo atmosférico en el momento del avistamiento clasificado según un sistema de categorías más simple que el empleado en weather (fog, clear-night, partly-cloudy-night, partly-cloudy-day, cloudy, clear-day, rain, wind).

\item \textbf{sunriseHour, sunriseMinute, sunsentHour y sunsetMinute}: hora y minuto local en que amaneció o atardeció.

\item \textbf{sunsetMinutesMidnight,sunriseMinutesMidnight}: minutos tras las 00:00 en los que amaneció o atardeció.

\item \textbf{sunsetMinutesBefore,sunriseMinutesSince}: según la página del dataset es los minutos del avistamiento relativos al amanecer y al atardecer.
 
\item \textbf{population_density}: densidad de población por km^2 en un avistamiento.

\item \textbf{urbal-rural}: cómo de urbana es la localización donde apareció el pokemon relativa a la population density


\end{itemize}